\section{Proof Sytem in zkSharding}
\label{sec:zkp}
Our zkEVM \cite{zkevm} operates at the bytecode level by directly
interpreting EVM bytecode. It  ensures high compatibility with existing
Ethereum dApps and smart contracts, although it may produce slightly
different state roots than the standard EVM due to the use of
SNARK-friendly optimizations. Projects like Scroll \cite{scroll} and
Hermez \cite{hermez} by Polygon also use this method.

In our zkEVM, we use FRI-based placeholder proof system \cite{placeholder}
which uses  lookup argument \cite{plookup}.  During the implementation of
Plookup and its practical use, we encountered some technical issues that
were not mentioned in the original solution. Therefore, we propose
practical improvements \cite{plookuptweaks} for writing large PLONK
circuits with a complex logic.

zkSharding’s zkEVM consists of multiple subcircuits\todois{ref to PSE
	solution}, each of which is handled by separate provers. A
straightforward
approach would be to have each prover $\prover_i$ generate an independent
FRI-based proof $\pi_i$ for each subcircuit $\circuit_i$, resulting in $M$
proofs (where $M$ is the number of subcircuits). These proofs would then
need to be aggregated into a succinct proof $\pi$. However, FRI lacks
homomorphic properties, making this aggregation process computationally
expensive, which contradicts FRI's main advantage of prover efficiency.

To address this, we designed Distributed FRI (DFRI) \cite{dfri}, which
uses FRI batching techniques to enable efficient proof aggregation across
multiple provers. In DFRI, the committed polynomials from each prover are
combined using a collaboratively generated random challenge. This batching
mechanism allows us to aggregate the proofs more efficiently while
maintaining the security and integrity of the proof system. DFRI also
ensures accountability by making dishonest behavior from any prover
detectable, which is crucial for maintaining liveness in distributed
systems like zkSharding.

From a performance perspective, DFRI maintains the same proof size as a
single FRI proof, as compared to the straightforward approach which would
involve having $M$ proofs without infeasable aggregation layer. This is
achieved through coordination between provers during the proving process
by slightly increasing the communication overhead. Additonally, DFRI does
not extend the overall proving time comparing to the single FRI-based
prover.
Overall, DFRI helps retain the efficiency of FRI while enabling secure,
distributed proof generation, making it a critical innovation for
zkSharding's scalability and security.