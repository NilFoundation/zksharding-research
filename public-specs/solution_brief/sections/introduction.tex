\section{Overview}
\label{sec:introduction}

\todoisinline{We need additional info on: async programming model, refund
	mechanism \handan{DONE: I integrated async model to in Section
		\ref{sec:introduction}, \ref{sec:executionshards} and
		\ref{sec:txlifecycle}. I also mention failed tx when describing the block
		creation in Section \ref{sec:txlifecycle}.}}

=nil;’s zkSharding introduces a scalable sharded-zkRollup
solution for Ethereum that leverages
sharding and SNARKs to address the network’s scalability challenges while
maintaining decentralization and security. Unlike conventional rollups,
zkSharding partitions the network into parallel execution shards, where
each shard processes transactions independently while maintaining unified
liquidity and state. The contracts deployed on these shards communicate
\emph{asynchronously} \cite{asyncNil}, enabling them to send messages to
contracts on other shards without pausing execution to await the message
results.
This approach allows zkSharding to achieve
significantly higher throughput without fragmenting liquidity or
increasing the complexity of cross-shard interactions. In this document,
we explain how we achieve these objectives in detail.

At a high level, zkSharding is composed of three interconnected
components. These components are not structured as hierarchical layers,
but rather work collaboratively, each serving a distinct role:

\begin{enumerate}
	\item\textbf{ L1 (Ethereum):} Acts as the settlement and data
	availability layer. Shards submit aggregated state proofs and data
	commitments
	here for finalization and security.
	\item\textbf{ Main Shard:} Coordinates execution shards, manages
	cross-shard communication, and ensures system-wide state
	synchronization
	and integrity.
	\item \textbf{Execution Shards:} Handle user transactions and
	      smart contracts in parallel, each maintaining its own state
	      and processing
	      a subset of transactions.
\end{enumerate}


zkSharding’s primary objective is to enable scalable computation within a
decentralized framework. By dividing workloads across multiple shards,
zkSharding increases transaction throughput without centralizing control.
L1 acts as an additional layer of security by verifying zkSharding’s state
transition validity through submitted validity proofs. These proofs allow
zkSharding to integrate with Ethereum’s canonical blockchain. In this way,
it ensures efficiency and security while preserving decentralization.
Next, we give an overview of the role of L1 in zkSharding system before
diving into the design of zkSharding (See Figure \ref{fig:blackbox}).