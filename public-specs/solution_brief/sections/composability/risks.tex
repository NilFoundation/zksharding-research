\section{External transactions associated risks}

For any novel and unique solution, a crucial aspect involves conducting a thorough risk 
assessment, which should be articulated alongside the proposed solutions. In this section, we 
delve into some key risks that demand consideration concerning the technology.


\subsection{Inconsistency of Ethereum transactions}

The term inconsistency means the conflicts of transactions on the L1 network from the same user. 
The risk of inconsistency among Ethereum transactions is substantial, primarily because proposers 
face challenges in accurately determining the latest nonce of an externally owned account (EOA). 
The options available, such as obtaining it from an internal Ethereum light client or including 
it in the nil's transaction, both carry inherent risks. Users must be mindful of providing the 
most up-to-date nonce, and they must refrain from committing transactions on Ethereum before 
transactions from nil;'s cluster are submitted to L1. Failing to observe this sequence may lead 
to inconsistencies in transactions.

Furthermore, in both models, users are unable to assess the exact number of Ethereum 
transactions that will be submitted to L1 from L2 with their signature, making it unpredictable 
to determine the nonce value for the standalone user transaction in advance. The occurrence of 
duplicated nonces is problematic, as a transaction with identical or lower nonce is deemed 
incorrect and subsequently not processed. This becomes especially precarious when the rejected 
transaction is one originating from the cluster.


\subsection{High final cost for users}

The ultimate cost of Ethereum transactions can become a significant financial loss for users, 
primarily due to the necessity of setting a gas price higher than the current on-market rate. 
This precaution is essential because the sequencer may decline to include the transaction in a 
block, risking rejection from the builder if the market price rises more than one is set in 
a transaction. As the delay from signing Ethereum transaction to pushing it to the builder may 
be long enough, a higher gas price needs to be established in advance, thereby inflating the 
final cost for users. This cost increese is particularly high in the preprocessing model, where 
Ethereum transactions are duplicated.


\subsection{Uncomputable timeframes}

In both design cases, the timeframe between sending a nil's transaction and making a call on the 
Ethereum target application is entirely unpredictable. Several significant factors contribute to 
this uncertainty. Firstly, batched transactions are not guaranteed to be included in the 
subsequent sequencer block. Secondly, the timing of when a block will be accepted by the relayer 
or builder (depending on the sequencer model) is uncertain. In the case of the on-demand design, 
this timeframe also encompasses the time required for user signing and transaction verification.

These factors collectively imply that despite the relative predictability of the finalization of 
the cluster, the finalization of external transactions is unfeasible to precompute.


\subsection{Unrevertable transactions}

In the current design, achieving full atomicity is not feasible. Consequently, there exists a 
risk wherein the nil;'s cluster state may undergo reversion while the transaction on Ethereum 
successfully executes. This vulnerability is particularly critical in the preprocess model, 
where a validator could potentially exploit maliciousdata as input for a user transaction signed 
in advance. Even if the state of the nil;'s cluster is reverted, the malitius data and call 
transactions on L1 can still proceed successfully, posing a substantial risk to users, 
especially in scenarios where the execution of a malicious transaction on Ethereum prevails over 
slashing risks. The on-demand model mitigates this vulnerability by allowing users to await 
block finalization on the cluster before signing the proposed transaction, thus circumventing 
the mentioned pitfall.
