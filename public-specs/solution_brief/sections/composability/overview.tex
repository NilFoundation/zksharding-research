\section{Overview}
\label{section:overview}

Both presented approaches are concluded with their essential advantages and disadvantages. 
That should be mentioned the technical feasibility was confirmed on the artificial simulation of 
the solutions.


\subsection{External transactions on-demand}

A high-level and step-by-step description of the processing of external transactions 
by demand is presented below (Figure 1).
\begin{figure}[H]
    \centering
	\includegraphics[scale=0.8]{figures/on_demand_design.pdf}
    \caption{On-demand transactions processing design}
     \label{figure:on-demand-design}
\end{figure}
We cover the most comprehensive case that will trigger the sub-sceneries with other external 
transaction types.

\begin{enumerate}
    \item A user creates an external transaction on the nil;'s cluster. That transaction 
    invokes a function marked with extcall, from the contractnatively deployed on the cluster. 
    Formally, the transaction may look like this:

    \begin{verbatim}
        {
            chainID: <nil chain ID>,
            from: <user addr>,
            to: <nil 20-bytes contract address>,
            type: "extcall",
            ....
            data: <function selector>||<input params>
        }
    \end{verbatim}

    \item The transaction is sent to the mempool;
    \item Builders on their own or with the help of searchers derive and batch transactions 
    from mempool. Then batch is sent to the proposer by auction or with the help of 
    the relayers;
    \item Proposer validates the transaction, and there can be several cases:
        \begin{enumerate}
            \item The transaction is marked as native -- no Ethereum transaction will be generated;
            \item The extcall transaction -- that is expected that there will be generated 
            some number of Ethereum transactions. User Ethereum nonce should be provided;
            \item Ethereum address as the trarget (field "to") -- there will be generated a 
            transaction on Ethereum. User Ethereum nonce and transaction signature should be 
            provided. Note that it's not mandatory for the user to provide the complete 
            transaction, as RLP-serialized data can be easily reproduced from some basic 
            metainfo and data from the user that will reduce the cost of native transactions, 
            while validators can construct transactions for cheap; 
            \item Batch of signed Ethereum transactions from the user -- the proposer is not 
            required to perform any specific actions but rather to propagate the ordered batch 
            to the sequencer.
        \end{enumerate}
    \item For the extcall transaction, each time during validation proposer detects 21-bytes 
    address, it creates a special record in temporary memory, that may
    contain: new nonce, input data, target contract, function selector, callback address 
    (if exists);
    \item When native block validation is completed, the proposer sends to the signature pool a 
    specially composed structure with prepared Ethereum transactions generated during validation 
    process;
    \item The user identified block confirmation containing their external transaction and by 
    request gets a batch of Ethereum transactions from the signature pool;
    \item There are several options to ensure the correctness of input data in L1 transactions. 
    However, the simplest and most dependable method involves compelling other 
    attestators to verify the Ethereum transaction data either. Through this process, users can 
    assess the number of signatures from the block committee that have endorsed 
    the batched transactions;
    \item The user signs Ethereum transactions from the signature pool and constructs a new 
    native transaction, consolidating L1 transactions into a batch within it. 
    Subsequently, this transaction is sent to the mempool;
    \item Upon detection of such a transaction, the proposer straightforwardly propagates the 
    batch to the sequencer.
\end{enumerate}

\subsubsection{Client application}

There will be specific changes in the client application, primarily evident in two distinct 
modules for signatures and transaction construction: one designed for Ethereum and the other 
tailored for nil;'s transactions. Even if the signature standard remains consistent, variations 
in the transaction signing process and content may arise.

Another crucial modification involves the verification module. The client application must 
verify the correctness of input data for Ethereum transactions generated by the proposer during 
the validation process of an extcall. The Ethereum transaction from the proposer should undergo 
verification before being signed to mitigate the risk of fraudulent activities. 

In addition, it makes sense for the user verification mechanism to await the finalization of the 
corresponding block. This precaution is warranted due to the absence of atomicity in 
transactions, meaning that even if a rollback occurs in nil;'s cluster, the transactions 
submitted on Ethereum may not be reverted. Consequently, waiting for block finalization adds 
an extra layer of assurance in maintaining consistency and reliability in user verification 
processes.

\subsubsection{Signature pool}

The signature pool is a temporary storage for the records of Ethereum transactions that may 
look like:
\begin{verbatim}
    H(<nil address>||<external transaction hash>) : [
        txs: {eTx1', eTx2', ...}, // non user-signed transactions -> (r, s) = (0, 0)
        sig: <aggregated signature of the committee>,
        preserve_commitment: ...,
        block_id: ....
    ]
\end{verbatim}

preserve\_commitment (prc) is a special field, that will allow removing record only when a user 
signs the transactions and send an aggregated external transaction.
It can be constructed in the following way: 
\[ prc=H(H(txs)||(user_{commitment})), \] 
where \[ user_{commitment} = H((Pb \times rand_{nonce})) \]

In this approach, the removal of an entity from the pool occurs only when the user signs the 
transactions and sends them back in the same order within the aggregated transaction. 
Additionally, the user includes their \( rand_{nonce} \), confirming their intention to 
eliminate the record.

There is no necessity for a separate network pool to store these records, as they can 
technically be housed in the same area as the transaction pool. It is evident that the load on 
the mempool will not significantly increase, given the relatively small number of Ethereum 
transactions. Moreover, temporary storage costs are incurred to prompt users to sign 
transactions from the pool. Both these considerations suggest that a standard native mempool 
can effectively be employed to manage the signature pool.

\subsubsection{Tradeoffs}
Key advantages of this design include:
\begin{enumerate}
    \item Ethereum transactions order preservation -- enabling the development of sophisticated 
    solutions. As an illustration, complex operations such as transferring funds from Ethereum 
    to \nil cluster can be seamlessly executed with just a single function call;
    \item Arbitrary function calls on L1 -- as users sign transactions explicitly, on the 
    Ethereum side everything will look as if would user directly creates and sends it;
    \item Simple implementation -- the design doesn't require difficult decisions or 
    architecture, while the implementation of the client side will not be overcomplicated;
    \item An unlimited number of Ethereum transactions per one extcall function -- as validator 
    creates transactions and increases nonce, the potential number of Ethereum transactions 
    limited only by native gas limit.
\end{enumerate}
\begin{enumerate}
    \item Active client application participation -- is not as significant a concern in practice 
    as it might initially appear, particularly for wallets and clients other than hardware 
    devices. In any scenario, transactions are retained in the signature pool until the proposer 
    confirms the submission to L1;
    \item Separation of pools -- despite the fact that division can be done virtually it may require 
    additional development efforts to support.
\end{enumerate}
\subsection{External transactions pre-process}
A high-level and step-by-step description of pre-processed transactions is presented below 
(Figure 2).

\begin{figure}[H]
    \centering
	\includegraphics[scale=0.8]{figures/preprocess_design.pdf}
    \caption{Pre-processed transactions design}
     \label{figure:staking-market}
\end{figure}

The approach is called transaction pre-processing, as additional effort will be required from 
the user side. Below, we cover the scenario:
\begin{enumerate}
    \item User generate an extcall transaction;
    \item The client is required to preprocess a sequence of consecutively signed Ethereum 
    transactions destined for a predetermined address;
    This address is determined in advance, often based on the anticipated execution flow, 
    such as using an execution flow tree;
    \item Builders on their own or with the help of searchers derive and batch transactions 
    from mempool. Then batch is sent to the proposer by auction or with the help of the relayers;
    \item Proposer validates the transaction. Each time validator detects extcall, it creates a 
    special data transaction on Ethereum and "binds" (by ordering) this transaction with the 
    corresponding L1 call transaction. In the Figure, it is shown as (eTx\_d, eTx);
    \item Batched Ethereum transaction sent to sequencer;
    \item Sequencer aggregates Ethereum batch and submits it to L1 builders;
    \item Ethereum proposers validate transactions, starting with the data transaction. 
    In this process, the transaction puts data into the corresponding proxy contract;
    \item Following the data transaction, the call transaction with the user's signature will be 
    processed in sequence. This call is directed to the dedicated proxy where the data was 
    previously saved;
    \item The proxy contract function employs a call or delegatecall to the target contract.
    This function is invoked with the data stored earlier by the data transaction, signed with 
    the committee's signature.
\end{enumerate}


Before submitting a native transaction with an external call, the client must prepare a set of 
consecutive signed Ethereum transactions. The addresses for these transactions will be derived 
from the Control Flow Processor (CFP) from the source function. It is important to note that the 
use of extcall is not allowed under explicit branching statements (recursion, loops, conditions). 
Despite the fact that a revert implies an implicit flow break, it is still permitted within the 
main flow. The CFP will disregard implicit main flow branching. The overall approach enables the 
pre-construction of Ethereum transactions without obvious reduction, ensuring proper 
identification of the target address.

It is essential to note that changes to the proxy address must occur transparently without 
immediate replacement. This is because the proxy address is derived during the Ethereum state 
\( S_i \), while the actual proxy call takes place during state \( S_j \), where i<j. In the 
need of an upgrade of the proxy contract on Ethereum, the address on Nil's cluster will need to 
be updated after L1 finalization. Complete replacement on Nil will occur only after the 
finalization of the cluster. "state\_source" stated before the last completed replacement will 
not be accepted.

The CFP will determine the number of Ethereum transactions to be prepared and the target address 
for each of them. The client constructs a batch of sequenced transactions with substituted 
addresses and signs them. Subsequently, the client prepares a joined native transaction that 
may appear as follows:

\begin{verbatim}
    {
        chainID: <nil chain ID>,
        from: <user addr>,
        to: <nil 20-bytes contract address>,
        type: "extcall",
        ....
        data: <function selector>||<input params>
        state_source: <blockID>, // state from which the proxy address derived
        eth_tx: [
            eTx1, eTx2, ..., eTxn
        ]
    }
\end{verbatim}


\subsubsection{Data transaction}
In order to sign transactions on the L1, a data transaction must precede and store data on the 
proxy contract. This prerequisite arises because, for the transaction to be signed, data must 
be known in advance. The challenge arises from the fact that the input data is only 
ascertainable when the nil;'s contract invokes an external call during the execution process. 

To overcome this limitation, the proposer initiates a new Ethereum transaction, termed a 
"data transaction". Within this transaction, a specific function on the proxy contract, such 
as \(updateData\) is invoked with already computed extcall data as a parameter. Subsequently, 
the proposer batches the data transaction along with the corresponding call transaction, 
ensuring that the update data call to the proxy occurs immediately before the data update. 
Through this approach, clients can proactively sign Ethereum transactions in advance, even 
without knowledge of the actual input parameters.


\subsubsection{Proxy contract}

Proxy contracts play a pivotal role in the execution of user-signed transactions by utilizing 
data previously saved by the proposers' data transaction. There are essentially two approaches 
to invoke target contracts from proxy: through the use of \(delegatecall\) and \(call\). When 
employing \(delegatecall\), the target contract is executed within the context of the proxy 
contract, with the proxy credentials remaining unaltered. Conversely, the \(call\) method allows 
the target contract to operate within its own context, while the sender credentials undergo a 
transition from an externally owned account (EOA) to those of the proxy. The most important 
credentials for such calls one should take into account are \(msg.sender\) and \(msg.value\).

It is noteworthy that, as of the present, there exists no technical mechanism to invoke 
functions from another contract while preserving the target context without making modifications 
to the sender data. That means there must be made a choice between preserving own credentials or 
the target context within a call.

The code snippet below is an illustration of a proxy contract utilizing a delegatecall. In the 
initial data transaction, information is stored in the internal storage under a specific key, 
where the key may simply be a hash of the user's address. Before data is saved, a validation 
check is performed to ascertain if the externally owned account (EOA) is a valid nil;'s validator 
and possesses the authorization to save data. Subsequently, a user-signed transaction triggers 
the "transfer" function. Upon retrieval of data from storage using the user's address, the 
function initiates a delegatecall to the USDC contract. It is crucial to note that this call 
does not alter the context of the target contract.

\begin{verbatim}
    contract USDC_proxy is IUSDC { 
         ... <original usdc contract data fields> ...
        mapping(bytes32 => bytes) input_storage;

        //uint256 -- blockID after which the validator right expires.
        // if < curBlockID -- update reverted
        mapping (address => uint256) nil_validators;  

        address usdc;

        function upateData(bytes32 key, bytes calldata _input) public {
            revert(ifUpdateNotAllower(msg.sender)); // Optionaly
            input_storage[key] = _input;
        }

        function transfer() {
            bytes memory raw_input = input_storage[keccak256(msg.sender)];
            (... input ...) = parseBytes(raw_input);
                    (bool success, bytes memory data) = usdc.delegatecall(
            abi.encodeWithSignature("<abi>", ... input ...); // or call/staticcall
        }
    }
\end{verbatim}



\subsubsection{Tradeoffs}
\begin{enumerate}
    \item Potentially faster than on-demand schema, because you won't need to perform nil's 
    double transaction;
    \item More robust liveness guarantees as user is not involved;
    \item Complex client -- it requires advanced CFP to precisely compute the number of extcall 
    that will raise during execution;
    \item Proxy contracts maintaining -- all proxy contracts will have to be properly managed to 
    be up to date with relevant target addresses and validators;
    \item Double Ethereum transactions -- for each call transaction, a dedicated data 
    transaction is needed, which increases the final cost for users;
    \item Proxy-contracts -- target contracts require the implementation of dedicated proxies, 
    thereby introducing heightened maintenance and development expenses;
    Utilizing any arbitrary Ethereum contract without a proxy is not feasible under this design;
    \item Limited extcalls -- no way to utilize extcalls with branching, which limits 
    development flexibility and comprehensiveness of solutions;
    \item Credentials limitation -- any target contract whose logic is bound to \(msg.sender\) 
    rather than \(tx.origin\) will not work properly;
    \item Risks of malicious behavior -- as the raw transactions are publicly available, malicious
    actors can submit them directly to Ethereum before the protocol commits it or sends data transactions.
    Despite not providing the actor with direct income gain -- it has reputational and protocol risks. 

\end{enumerate}



\subsection{Extacall feedback model}
In order to obtain and process results from an extcall, an asynchronous callback mechanism can 
be employed. Contracts within the nil network that incorporate extcall functionality include 
specific functions designed to handle raw results from the target contract on Ethereum. Upon 
deployment, such a nil;'s contract initializes a lookup table for specific callback handlers, 
each identifiable by a simple ID.

When an extcall is expected to return a result or trigger an event, the contract explicitly 
initiates a special event, as demonstrated in the code snippet below. Subsequently, the 
EthDataProvider filters feedback data from the Ethereum network and invokes the appropriate 
feedback function within the nil cluster. An illustrative example of such a nil;'s application 
is presented below:

\begin{verbatim}
    struct Handler {
        address handler_address;
        string abi;
    };

    mapping (bytes32=>Handler) lookup;

    function someFunc(address addr) public extcall {
        addr.call(....); // extcall that is expected to return result to handle
        emit FeedbackRequest(
            address(this), // source contract
            keccak256(handler_address, abi), // handler id
            keccak256(addr, userEthAddress, userNonce) // unique id of extcall by which to filter
            );
    }

    function processResult(bytes32 handlerId, bytes callback data) {
        lookup[handlerId].call(lookup[handlerId].abi, data);
    }
\end{verbatim}
