\section{Foundations for \protocol}
\label{section:preliminaries}

This section describes the precursor set of technologies leading to the
product description in Section \ref{section:proposal}.

\subsection{zkLLVM}

zkLLVM (\cite{zkllvm}) is a circuit compiler designed to translate high-level 
mainstream languages, such as C++ and Rust, into representations suitable for 
provable computation protocols, namely circuits for proof systems.

The architecture of zkLLVM offers distinct advantages over other 
methods of circuit development:
\begin{enumerate}
    \item It allows users to directly compile their algorithm,
        without needing a custom Domain Specific Language (DSL) 
        and duplicating source code.
    \item It omits any intermediary layer, such as a specific zkVM, 
        between the original algorithm and the resulting circuits. 
        This absence translates to no additional overhead in the circuit size 
        (and consequently, the proving time).
    \item Due to direct access to the inner circuit representation, zkLLVM facilitates 
        the generation of optimized low-level verifier code tailored for 
        specific virtual machines. 
        For example, \nil utilizes the zkLLVM transpiler\footnote{
            \url{https://github.com/NilFoundation/zkllvm-transpiler}} 
        for the EVM Placeholder verifier.
    \item As an LLVM-based compiler, zkLLVM boasts compatibility with any LLVM 
        IR-based extension.
\end{enumerate} 

zkLLVM assumes a crucial role in the novel zkEVM construction approach, 
detailed further in Section \ref{section:zkvm}.


\subsection{Placeholder Proof System}

Initiated in 2021, Placeholder \cite{placeholder} is 
a modular IVC (Incrementally Verifiable Computation) that relies 
on a PLONK-inspired arithmetization-based proof system. 
Its inherent modularity provides the flexibility for various adjustments based on the specific use case:

\begin{itemize}
    \item \textbf{Underlying Fields}: 
        For algorithms that do not involve cryptographic operations, 
        there is often no requirement for 256-bit fields. 
        To optimize both the proving and verification processes, 
        Placeholder can be adapted to work with 64-bit fields.
    \item \textbf{Commitment Schemes}:
        The choice between commitment schemes, such as 
        those based on hashes or bilinear pairings, allows 
        for trade-offs between the necessity for a trusted setup 
        and associated verification expenses.
    \item \textbf{Lookup Techniques}:
        Different techniques, like Plookup or Baloo, can be employed 
        depending on the size of the lookup table and 
        the frequency of lookup requests, thus reducing the lookup overhead.
    \item \textbf{Gate Generation Techniques}:
        Efficient IVC is achieved using an approach similar to 
        Protostar for gate definition. Nonetheless, this might 
        not always be the optimal choice, especially when there is 
        no need for recursive proof verification.
\end{itemize}

Such modularity enables Placeholder to reduce the confirmation times 
on the L1-layer and speed up zkBridge data provision. 
This is achieved through faster IVC-based zkVM 
and Ethereum consensus proof generation.


\subsection{zkBridge}

zkBridge is a concept that emerged in early 2021 from collaborative efforts 
between the \nil Foundation, Ethereum Foundation, and Mina Foundation 
\cite{evm-mina-verification-design}. 
Later in the same year, the Solana Foundation joined this initiative 
\cite{evm-solana-verification-design}.

This concept materialized into a series of trustless bridging 
 projects amongst the Mina, Ethereum, and Solana networks. 

The data access approach detailed in Section \ref{section:eth-data-access} 
harnesses the zkBridge concept to minimize trust assumptions.

\subsection{Crypto3}

Crypto3\footnote{\url{https://github.com/NilFoundation/crypto3}} 
 is a modular cryptographic suite developed in C++17. 
It is designed to support the research 
 and development of innovative primitives and protocols
 by creating new constructions or combining existing ones. 
The suite includes a variety of constructions, such as basic algebraic structures 
 (like finite fields and elliptic curves) and cryptographic primitives 
 (for instance, symmetric and asymmetric encryption,
  threshold signatures, and algebraic hash functions).

Additionally, Crypto3 serves as a C++ Software Development Kit (SDK) for zkApp development, 
making it easier to use zkLLVM. 
For example, zkBridge solution developed with \nil technologies utilizes Crypto3 
as the SDK for zkLLVM.
 
\subsection{Proof Market}
\label{section:proof-market}

Proof Market\footnote{\url{proof.market}} is a decentralized platform. 
It is designed to outsource zkProof generation tasks. The foundational premise 
behind Proof Market is straightforward. It aims to aggregate the hardware-intensive 
demand for proof generation from numerous protocols. Then, it reallocates these 
tasks to independent entities. These entities include professional hardware operators, 
equipment owners, and circuit/hardware designers.

This arrangement fosters a dynamic marketplace. In this marketplace, hardware 
operators can directly cater to the needs of proof consumers. Proof Market can 
maintain a consistent and robust demand due to its substantial user base. This 
sustained demand draws in large-scale hardware operators. They are keen on 
maximizing the utilization of their resources.

Proof Market presents a competitive landscape for proof producers. This environment 
compels them to relentlessly pursue the optimization of proof generation overheads.
Several strategies can achieve this. Producers might scale up their infrastructure.
This scaling can benefit from economies of scale. Alternatively, they can refine 
their hardware for optimal proof generation efficiency. A producer who offers 
competitive pricing and rapid proof generation can dominate the market. This 
dominance incentivizes other operators to enhance their competitive offerings.

By leveraging Proof Market for proof generation, \protocol achieves a significant 
advantage. It can ensure decentralized proof generation from the day one. At the 
same time, it distinguishes between the roles of validators and proof generators.
