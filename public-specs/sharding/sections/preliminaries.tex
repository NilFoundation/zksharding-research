\section{Preliminaries}
\label{section:preliminaries}

\paragraph{Network Model.}
The system operates under a \textit{partial synchrony} model.
In this model, after an unknown Global Stabilization Time (GST), 
 the network achieves synchrony with a known maximum delay $\Delta$.
This approach recognizes that synchrony might be temporarily disrupted, 
 potentially due to attacks, but it is expected to eventually stabilize. 
 A distinction is made between the maximal network delay $\Delta$ post-GST 
  for worst-case scenarios and an actual network delay $\delta$ 
  for average or optimistic case scenarios.

\paragraph{Adversary Model.}
It is assumed that up to $f$ of the shard's committee members are malicious.
Hence, the committee size $n$ is at least $3f+1$.
The adversary can diverge from the specified protocol in any way.

\paragraph{Protocol Properties.}
A protocol has \emph{optimistic responsiveness} if in an optimistic case it takes 
 $O(\delta)$ to make a decision.
In other words, protocol operates at the speed of the network.

A protocol is considered \emph{safe} if at all times, for every pair of correct nodes, 
 the output log of one is a prefix of the other.

A protocol provides \emph{liveness} if, after GST, all non-faulty nodes repeatedly output growing logs.

\paragraph{Consensus Algorithm Background.}

A \emph{view} in consensus protocols refers to a specific configuration 
 or state of the network.
The protocol operates in a sequence of \emph{views}, 
 where each view has a designated leader.

The protocol is decomposed into two subprotocols:
  \emph{view-synchronization} and \emph{in-view operation}. 
The view-synchronization subprotocol, also called \emph{pacemaker},
 is used by parties to enter a new view 
 and spend a certain amount of time in the view.
The \emph{in-view operation} subprotocol is used by parties to commit a block.
This decomposition, which is used by HotStuff \cite{HotStuff} and its successors, 
 allows us to analyze \emph{safety} and \emph{liveness} properties of the protocol separately.

\paragraph{Proposer-Builder Separation (PBS).}
 This framework divides the role of single validator into two roles: proposer and builder. 
 Block builders are responsible for constructing the actual contents of a block, including 
 ordering and veriffication transactions. Block proposers are responsible for proposing 
 (validation and propogation) new blocks to be added to the network. 

 \subsection{Multi-Threshold BFT}
 \label{section:preliminaries:multi-threshold-bft}
 
 It is expected to have in partially synchronous systems more emphasis on safety than liveness:
 The system is designed to be safe always, while liveness is guaranteed only after a certain time.
 Moreover, both attacks on safety and liveness require committee reformations, but the former also requires a state reformation.
 The inconvenience is that popular BFT SMR protocols, such as PBFT \cite{PBFT}, Tendermint \cite{Tendermint}, and HotStuff \cite{HotStuff},
 have the same threshold for safety and liveness, a third of the committee size.
 However, it is possible to decouple the safety and liveness thresholds \cite{MultiThresholdBFT}.
This section describes some properties of the Multi-Threshold BFT protocols.
   
 The analysis of the work of \cite{MultiThresholdBFT} provides a framework 
  to design and update protocols to have optimal safety a
  nd liveness thresholds in both partially synchronous and synchronous settings. 
For the purposes of this paper, attention is focused on the partially synchronous setting, 
and a brief summary of the relevant results are provided.
 
 \begin{itemize}
     \item In the partially synchronous model there exists a BFT SMR protocol with a \emph{safety threshold} of \(f_s \geq n/3 \) and a \emph{liveness threshold} of \(f_l\), that
     must satisfy just the following condition:
     $$
     f_l \leq \frac{n - f_s}{2}.
     $$
     \item The protocol is based on a Sync HotStuff protocol \cite{SyncHotStuff}, therefore it has the same 2-vote structure.
     There are two main differences:
     \begin{itemize}
         \item Different quorum size, necessary to create a Quorum Certificate, which is equal to \(n - f_l\).
         \item The protocol is not optimistically responsive (however, for a local consensus, where committee size is not large, it should not become a problem, based on the performance evaluation in \cite{SyncHotStuff}).
     \end{itemize}
 \end{itemize}
 
 \begin{remark}
     By giving up responsiveness, the protocol can become significantly more safe. For example, the liveness threshold can be set to $n/4$,
     while having a safety threshold of $n/2$, exactly like in the synchronous setting.
     This improvement is crucial for the sharding protocol, where shards' safety threshold should be strictly greater than the safety threshold of the whole cluster.
     It is discussed in more detail in Section~\ref{section:sharding}.
 \end{remark}